\chapter{Proyecto}

\section{Objetivo General del Proyecto}
Objetivo general del proyecto

\section{Objetivos Específicos del Proyecto}
\begin{enumerate}
	\item Objetivo específico
	\item Objetivo específico
	\item Objetivo específico
\end{enumerate}

\section{Metodología de Desarrollo}
Para poder definir la metodología de desarrollo a utilizar, primero se debe tener en cuenta la \autoref{table:risks}, que representa los riesgos asociados.

\begin{center}
	\begin{tabular}{|lll|l|l|ll|}
		\hline
		\multicolumn{7}{|c|}{\multirow{7}{*}\textbf{\textbf{Tabla de Riesgos}}} \\
		\hline
		\multicolumn{3}{|l|}{\multirow{2}{*}{\textbf{Experiencia en el Problema}}} &
		Alta &
		X &
		\multicolumn{2}{l|}{\multirow{2}{*}{\begin{tabular}[c]{@{}l@{}}Se tienen años de experiencia con la \\empresa.\end{tabular}}} \\ \cline{4-5}
		\multicolumn{3}{|l|}{} &
		Baja &
		&
		\multicolumn{2}{l|}{} \\ \hline
		
		\multicolumn{3}{|l|}{\multirow{2}{*}{\textbf{Tamaño del Problema}}} &
		Grande &
		X &
		\multicolumn{2}{l|}{\multirow{2}{*}{\begin{tabular}[c]{@{}l@{}}La cantidad de funcionalidades a im-\\plementar es muy alta.\end{tabular}}} \\ \cline{4-5}
		\multicolumn{3}{|l|}{} &
		Pequeño &
		&
		\multicolumn{2}{l|}{} \\ \hline
	
		\multicolumn{3}{|l|}{\multirow{2}{*}{\textbf{Complejidad del Problema}}} &
		Complejo &
		X &
		\multicolumn{2}{l|}{\multirow{2}{*}{\begin{tabular}[c]{@{}l@{}}El sistema es difícil \\de comprender y manejar completamente.\end{tabular}}} \\ \cline{4-5}
		\multicolumn{3}{|l|}{} &
		Simple &
		&
		\multicolumn{2}{l|}{} \\ \hline
		
		\multicolumn{3}{|l|}{\multirow{2}{*}{\textbf{Tamaño del Software}}} &
		Grande &
		X &
		\multicolumn{2}{l|}{\multirow{2}{*}{\begin{tabular}[c]{@{}l@{}}El software a construir require muchas\\funcionalidades.\end{tabular}}} \\ \cline{4-5}
		\multicolumn{3}{|l|}{} &
		Pequeño &
		&
		\multicolumn{2}{l|}{} \\ \hline
		
		\multicolumn{3}{|l|}{\multirow{2}{*}{\textbf{Complejidad Software}}} &
		Complejo &
		X &
		\multicolumn{2}{l|}{\multirow{2}{*}{\begin{tabular}[c]{@{}l@{}}El software debe implementar cálculos\\complejos (ratios, promedios, etc.).\end{tabular}}} \\ \cline{4-5}
		\multicolumn{3}{|l|}{} &
		Simple &
		&
		\multicolumn{2}{l|}{} \\ \hline
		
		\multicolumn{3}{|l|}{\multirow{2}{*}{\textbf{Experiencia Software}}} &
		Alta&
		X &
		\multicolumn{2}{l|}{\multirow{2}{*}{\begin{tabular}[c]{@{}l@{}}Se tiene una alta experiencia desarrollando\\software para la empresa.\end{tabular}}} \\ \cline{4-5}
		\multicolumn{3}{|l|}{} &
		Baja &
		&
		\multicolumn{2}{l|}{} \\ \hline
		
		\multicolumn{3}{|l|}{\multirow{2}{*}{\textbf{Modularidad Funcional}}} &
		Existe &
		X &
		\multicolumn{2}{l|}{\multirow{2}{*}{\begin{tabular}[c]{@{}l@{}}Las funcionalidades  pueden implementarse\\por separado y luego integrarse.\end{tabular}}} \\ \cline{4-5}
		\multicolumn{3}{|l|}{} &
		No existe&
		&
		\multicolumn{2}{l|}{} \\ \hline
	\end{tabular}%
  \\
  \captionof{table}{Tabla de Riesgos}\label{table:risks}
\end{center}

Aquí es recomendable escribir la interpretación de la tabla anterior y concluir por qué se eligió utilizar una determinada metodología.

\section{Técnicas y Notaciones}
\begin{itemize}
	\item Diagrama de Casos de Usos.
	\item BPMN para modelar el proceso de negocio actual.
	\item Carta Gantt para la planificación inicial del proyecto.
	\item Patrón de diseño MVC (Modelo, Vista, Controlador).
\end{itemize}

\section{Estándares de Documentación}
\begin{itemize}
	\item Adaptación Basada en IEEE Software Test Documentation Std 829-1998.
	\item Adaptación Basada en IEEE Software Requirements Specifications Std 830-1998.
\end{itemize}

\section{Software, Frameworks y  Lenguajes Utilizados}
\label{project:software}
A continuación se lista el software, frameworks y lenguajes de programación, marcado y estilos utilizados para la realización de este proyecto.

Para efectos del siguiente listado, los nombres de las herramientas, frameworks y lenguajes se han redactado en negrita, seguidos de paréntesis en itálica que contienen el número de la versión asociada a cada ítem.

\begin{itemize}
	\item[] \textbf{Lenguajes}
	\begin{itemize}
		\item \textbf{Ruby} \textit{(3.2.2)}: Lenguaje de programación de alto nivel.
		\item \textbf{HAML} \textit{(6.2.3)}: Lenguaje de marcado para la abstracción de HTML.
		\item \textbf{Sass} \textit{(6.0)}: Lenguaje de extensión para CSS.
	\end{itemize}
\end{itemize}

\begin{itemize}
	\item[] \textbf{Software}
	\begin{itemize}
		\item \textbf{MongoDB} \textit{(7.0.3)}: Base de datos orientada a documentos JSON.
		\item \textbf{Redis} \textit{(7.0.12)}: Almacenamiento en memoria, utilizado para el caché de datos.
		\item \textbf{RubyMine} \textit{(2023.2.2)}: Entorno de desarrollo integrado especializado para el trabajo con aplicaciones en Ruby, específicamente para Ruby on Rails.
		\item \textbf{Rake} \textit{(13.1)}: Librería de Ruby para la definición de tareas interdependientes.
		\item \textbf{MongoDB Compass} \textit{(1.39.0)}: Visor para bases de datos de MongoDB.
		\item \textbf{RedisInsight} \textit{(2.30.0)}: Visor para el almacenamiento del caché en Redis.
		\item \textbf{Docker Desktop} \textit{(4.21.0)}: Visor y gestor de contenedores de Docker, en formato de aplicación de escritorio multiplataforma.
		\item \textbf{NodeJS} \textit{(16.13.0)}: Entorno de servidor multiplataforma utilizado para la conversión de archivos en runtime.
		\item \textbf{Yarn} \textit{(1.22.21)}: Gestor de paquetes para JavaScript.
		\item \textbf{Docker} \textit{(24.0.2)}: Tecnología que permite crear y utilizar contenedores. Para efectos de este proyecto, es utilizado con el fin de probar el software desarrollado en distribuciones de Linux determinadas.
		\item \textbf{Termius} \textit{(8.7.2)}: Cliente SSH.
		\item \textbf{Git/Git Bash} \textit{(2.34.1)}: Sistema de control de versiones.
		\item \textbf{Ubuntu LTS} \textit{(18.04.6)}: Subsistema de Linux para Windows.
	\end{itemize}
\end{itemize}

\begin{itemize}
	\item[] \textbf{Frameworks}
	\begin{itemize}
		\item \textbf{Ruby on Rails} \textit{(7.1)}: Framework para desarrollo de aplicaciones web fullstack.
		\item \textbf{Jekyll} \textit{(4.0.0)}: Framework para desarrollo de aplicaciones web estáticas escrito en Ruby.
		\item \textbf{Bootstrap} \textit{(4.4.1)}: Framework para la creación de estilos, manejo de elementos visuales y la responsividad en aplicaciones web.
	\end{itemize}
\end{itemize}
