\chapter{Requerimientos del Software}

\section{Límites}

\begin{itemize}
	\item El software no permitirá...
\end{itemize}

\section{Caracterización de los Usuarios}
Describir a quienes apunta este proyecto. Se caracterizan por:

\begin{itemize}
	\item Pertenencia a un grupo etario entre...
	\item ¿Competencias técnicas en uso de software similares?
	\item Familiaridad con...
\end{itemize}

\section{Objetivo General del Software}
Describir el objetivo general del software.

\subsection{Objetivos Específicos del Software}
\begin{itemize}
	\item Objetivo específico
	\item Objetivo específico
	\item Objetivo específico
\end{itemize}

\section{Requerimientos Funcionales del Software}
A continuación, en las tablas x a y, se definen los requerimientos funcionales del software.
\begin{center}
	\begin{tabular}{ | l | p{15cm} |}
		\hline
		\multicolumn{2}{|c|}{\textbf{Módulo de Ejemplo}} \\
		\hline
		\multicolumn{1}{|c|}{\textbf{Id}} & \multicolumn{1}{|c|}{\textbf{Descripción}} \\
		\hline
		{\textbf{RF\_01}} & La plataforma contará con un módulo de creación... \\ \hline

		{\textbf{RF\_02}} & La plataforma contará con un módulo de visualización... \\ \hline

		{\textbf{RF\_03}} & La plataforma contará con un módulo de edición de un... \\ \hline
		
		{\textbf{RF\_04}} & La plataforma contará con un módulo de eliminación de un... \\ \hline
		
		{\textbf{RF\_05}} & La plataforma contará con un módulo de visualización de un... \\ \hline
	\end{tabular}
  
  \captionof{table}{Tabla de requerimientos funcionales del módulo de ejemplo}\label{table:rf:ejemplo}
\end{center}

\section{Requerimientos No Funcionales del Software}
La presente sección hablará de los requerimientos no funcionales del software desarrollado. Todos los requerimientos no funcionales se relacionarán con uno o más atributos. Si un atributo aplica a un requerimiento no funcional, eso quiere decir que el requerimiento contribuye a la calidad del software desarrollado a través de ese atributo. Todos los atributos listados están basados en la norma ISO 25010. A continuación, se presentan los requerimientos no funcionales en las tablas x a y.

% API
\begin{center}
  \begin{tabular}{ | p{2cm}| p{8cm} | p{5cm} |}
    \hline
    \multicolumn{3}{|c|}{\textbf{RNF\_01 (EJEMPLO)}} \\
    \hline
    
    \multicolumn{1}{|p{2cm}|}{\textbf{Descripción}} & \multicolumn{2}{|p{13cm}|}{La plataforma contará con una API REST} \\ \hline
    
    \multicolumn{1}{|p{3.5cm}|}{\textbf{{Atributo}}} & \multicolumn{1}{|p{1.5cm}|}{\textbf{Aplica}} & \multicolumn{1}{|p{10cm}|}{\textbf{Especificación}} \\ \hline
    
    \multicolumn{1}{|p{3.5cm}|}{\nohyphens{Adecuación Funcional}} & \multicolumn{1}{|c|}{X} & \multicolumn{1}{|p{10cm}|}{La API contribuye a la corrección funcional, ya que facilita la obtención de datos precisos del sistema a terceros.} \\ \hline
    
    \multicolumn{1}{|p{3.5cm}|}{\nohyphens{Eficiencia de Desempeño}} & \multicolumn{1}{|c|}{} & \multicolumn{1}{|p{10cm}|}{} \\ \hline
    
    \multicolumn{1}{|p{3.5cm}|}{\nohyphens{Compatibilidad}} & \multicolumn{1}{|c|}{X} & \multicolumn{1}{|p{10cm}|}{La API contribuye a la coexistencia con otras piezas de software independientes, ya que permite a dicho software consumir información del sistema en tiempo real.} \\ \hline
    
    \multicolumn{1}{|p{3.5cm}|}{\nohyphens{Usabilidad}} & \multicolumn{1}{|c|}{} & \multicolumn{1}{|p{10cm}|}{} \\ \hline
    
    \multicolumn{1}{|p{3.5cm}|}{\nohyphens{Fiabilidad}} & \multicolumn{1}{|c|}{X} & \multicolumn{1}{|p{10cm}|}{La API contribuye a la madurez del software, ya que es gracias a ella que el sistema puede satisfacer las necesidades de los usuarios que consumen información del mismo.} \\ \hline
    
    \multicolumn{1}{|p{3.5cm}|}{\nohyphens{Seguridad}} & \multicolumn{1}{|c|}{X} & \multicolumn{1}{|p{10cm}|}{Gracias al diseño de la API, sólo se exponen endpoints de lectura, por lo que esta contribuye a la confidencialidad e integridad de la información.} \\ \hline
    
    \multicolumn{1}{|p{3.5cm}|}{\nohyphens{Mantenibilidad}} & \multicolumn{1}{|c|}{} & \multicolumn{1}{|p{10cm}|}{} \\ \hline
    
    \multicolumn{1}{|p{3.5cm}|}{\nohyphens{Portabilidad}} & \multicolumn{1}{|c|}{} & \multicolumn{1}{|p{10cm}|}{} \\

    \hline
  \end{tabular}

  \captionof{table}{Tabla de requerimiento no funcional de API}\label{table:rnf:api}
\end{center}

\section{Interfaces Internas de Salida}

\begin{center}
	\begin{tabular}{ | c | p{3.5cm} | p{10cm} |}
		\hline
		\textbf{Id} & {\textbf{Nombre}} & {\textbf{Detalle de Datos}} \\ \hline
		{\textbf{IN\_01}} & Modelo X & field\_1, field\_2, field\_3 \\ \hline
		{\textbf{IN\_02}} & Modelo Y &  field\_1, field\_2, field\_3 \\ \hline
	\end{tabular}

    \captionof{table}{Tabla de interfaces internas de salida}\label{table:interfaces:in}
\end{center}

\section{Interfaces Externas de Salida}

\begin{center}
	\begin{tabular}{ | c | p{2cm} | p{6.5cm} | p{4cm} |}
		\hline
		{\textbf{Id}} & 	{\textbf{Nombre}} & {\textbf{Detalle de Datos}} & {\textbf{Medio de Salida}} \\
		\hline
		{\textbf{OUT\_01}} & Modelo X & field\_1, field\_2, field\_3 & Pantalla \\ \hline
		{\textbf{OUT\_02}} & Modelo Y & field\_1, field\_2, field\_3 & Archivo PDF \\ \hline
	\end{tabular}
  
  \captionof{table}{Tabla de interfaces externas internas de salida}\label{table:interfaces:out}
\end{center}
